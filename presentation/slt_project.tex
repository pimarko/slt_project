\documentclass{beamer}
\usepackage[utf8]{inputenc}
\usepackage{pgfplots}
\pgfplotsset{compat=newest}
\usepackage{biblatex}
\bibliography{ref}
\usepackage{subcaption}
\captionsetup{compatibility=false}

% Set nicer beamer colors

\definecolor{black}{RGB}{0,0,0}
\definecolor{white}{RGB}{255,255,255}
\definecolor{msgray}{RGB}{120,120,120}
\definecolor{msblue}{RGB}{11,93,174}
\definecolor{msred}{RGB}{206,62,21}
\definecolor{msyellow}{RGB}{232,163,26}
\definecolor{msgreen}{RGB}{100,161,27}
\definecolor{mspurple}{RGB}{106,20,125}
\definecolor{mslightblue}{RGB}{59,175,236}
\definecolor{msdarkred}{RGB}{145,0,33}

\setbeamercolor{normal text}{fg = black}
\setbeamercolor{alerted text}{fg = msred}

\setbeamercolor{titlelike}{fg = msblue, bg = white}
\setbeamercolor{section in toc}{fg = msblue, bg = white}
\setbeamercolor{background canvas}{bg = white}

\setbeamercolor{itemize item}{fg = msblue}
\setbeamercolor{enumerate item}{fg = msblue}
\setbeamercolor{itemize subitem}{fg = msblue}
\setbeamercolor{enumerate subitem}{fg = msblue}
\setbeamercolor{caption name}{fg = msblue}

\usepackage{graphicx}
\usepackage{hyperref}
% \usepackage{subcaption}

% Alter beamer layout


\setbeamersize{text margin left = 3em, text margin right = 3em}

\setbeamertemplate{footline}[frame number]{}
\setbeamertemplate{navigation symbols}{}
\setbeamertemplate{frametitle}{\hspace*{-1.9em}\bfseries\insertframetitle\par}
\setbeamertemplate{bibliography item}{\insertbiblabel}

\renewcommand{\vec}{\mathbf}

% Presentation title

\title{\textbf{Segmentation of Neuron Bundles from Diffusion MRI}}
\subtitle{SLT Course Project 2016}
\author{ Nico Previtali \\ Marko Pichler Trauber \\ Jakob Jakob}
\date{30. May 2016}

\institute[ETH Zürich]

\begin{document}

\begin{frame}
 \titlepage
\end{frame}


\begin{frame}
\frametitle{Presentation Outline}
\textbf{Overview} \newline
\begin{itemize}
	\item Model Extensions
%	\item Optimization
	\item Implementation
	\item Results
\end{itemize}
\end{frame}

\begin{frame}
\frametitle{Model Extensions I}
\textbf{Basis Model}: Data Clustering Using a Model Granular Magnet, Blatt et al. \cite{blatt1997data} \newline\newline
\textbf{Neighbourhood Definition}: No fixed $k$-nearest neighbourhood $\implies$ automatically select $k$ \emph{suitable} neighbours:
\linebreak
\begin{enumerate}
	\item Search for $k^*$-nearest neighbours of voxel $v_i$
	\item Select the $k$ neighbours $(k < k^*)$ with the most similar diffusion profile w.r.t to voxel $v_i$\linebreak
\end{enumerate}
We used $k^* = 26$ and $k = 6$ which empirically turned out to be suitable.
\end{frame}


\begin{frame}
\frametitle{Model Extensions II}
\textbf{Similarity Matrix}: \emph{Pairwise diffusion profiles similarities} instead of \emph{pairwise voxel distances}
\begin{equation*} 
	D_{ij} = 
	\begin{cases}
		\lVert s_i - s_j \rVert & \text{if $v_i$ and $v_j$ are mutual neighbours} \\
		0 & \text{otherwise}
	\end{cases}
\end{equation*}
\textbf{Use of Inner Products}: Further we looked at inner products $\langle s_i, s_j \rangle$ between the diffusion profiles $\implies$ allows to extend the model by applying a kernel. 
\begin{equation*} 
	D_{ij} = 
	\begin{cases}
		\langle s_i, s_j \rangle & \text{if $v_i$ and $v_j$ are mutual neighbours} \\
		0 & \text{otherwise}
	\end{cases}
\end{equation*}
\end{frame}


%\begin{frame}
%\frametitle{Optimization}
%Example: If you used Metropolis, mention it and maybe comment on your proposal distribution, but do not explain how it works in general. The same for deterministic annealing, etc. For any optimization strategy not discussed during the course, elaborate more.
%\end{frame}


\begin{frame}[fragile]
\frametitle{Implementation}
%State the software that you used. Did you use CPU, GPU, cluster, etc? Did you try to write efficient code? What performance differences did you observe? What were the most useful tricks that you applied? What else did you do to decrease runtime and memory usage?
\textbf{Language}: Python / numpy
\newline\newline CPU driven, single threaded implementation. Lots of nested loops and room to improve it.
\newline\newline \textbf{Attempted Approach}: First implement it correctly, then optimize $\implies$ We had hard times (even debugging with a small grid was very slow)
\newline\newline Unfortunately, no time left for optimizations like threads, GPU etc.
\end{frame}


\begin{frame}[fragile]
\frametitle{Results I}
\begin{figure}
\captionsetup[subfigure]{labelformat=empty}
\centering
\begin{subfigure}{0.45\textwidth}
  \centering
  \includegraphics[width=.8\linewidth]{fig/cluster_num.png}
  \caption{\#Clusters}%Relation between the total number of clusters and temperature with the selected temperature $T_{final}$.}
  \label{fig:cluster_num}
\end{subfigure}
\hspace*{\fill}
\begin{subfigure}{0.45\textwidth}
  \centering
  \includegraphics[width=.8\linewidth]{fig/magnetization.png}
  \caption{$\langle m \rangle$}%Relation between the magnetization and temperature.}
  \label{fig:magnetization}
\end{subfigure}
\begin{subfigure}{0.45\textwidth}
  \centering
  
  \includegraphics[width=.8\linewidth]{fig/suspencability.png}
  \caption{$\chi$}%Relation between the susceptibility and temperature.}
  \label{fig:suspenc}
\end{subfigure}
\hspace*{\fill}
\begin{subfigure}{0.45\textwidth}
  \centering
  \includegraphics[width=.8\linewidth]{fig/temp__.png}
  \caption{Distribution of $G$}
  \label{fig:gij}
\end{subfigure}
\end{figure}

\end{frame}



\begin{frame}
\frametitle{Results II}

\begin{figure}
\captionsetup[subfigure]{labelformat=empty}
\begin{subfigure}{0.45\textwidth}
  \centering
  \includegraphics[width=.8\linewidth]{fig/low_temp.jpg}
  \caption{Low Temperature}%L}
  \label{fig:low_tmp}
\end{subfigure}
\hspace*{\fill}
\begin{subfigure}{0.45\textwidth}
  \centering

  \includegraphics[width=.8\linewidth]{fig/high_tmp.jpg}
  \caption{High Temperature}%}
  \label{fig:high_temp}
\end{subfigure}

\begin{subfigure}{0.45\textwidth}
  \centering
  \includegraphics[width=.9\linewidth]{fig/figure_1.png}
  \caption{Super-paramagnetic Regime}%Final clustering result of a (12, 12, 12) voxel grid.}
  \label{fig:fig_good_final_cluster_12}
\end{subfigure}
\hspace*{\fill}
\begin{subfigure}{0.45\textwidth}
  \centering

  \includegraphics[width=.9\linewidth]{fig/cluster-8x8x8-2.png}
  \caption{Super-paramagnetic Regime}%Final clustering result of a (8, 8, 8) voxel grid.}
  \label{fig:fig_good_final_cluster_7}
\end{subfigure}

\end{figure}
\end{frame}


\begin{frame}
\frametitle{Results III}
\begin{center}
	\textbf{\Large{Video of Temperature Search /\\ Phase Transitions}}
\end{center}
\end{frame}

\begin{frame}
    \frametitle{References}
    \printbibliography
\end{frame}

\end{document}

